Face au volume croissant des données issues de la numérisation et de la ROC, des problématiques relatives à la qualité de ces données et à leur exploitabilité scientifique émergent, étant donné 
% émergent des problématiques relatives à la qualité de ces données et à leur exploitabilité scientifique étant données 
les erreurs dans les transcriptions de la ROC. Les scientifiques rencontrent ainsi des difficultés pour appliquer des outils informatiques, généralement entraînés sur des données textuelles correctement orthographiées \cite{DBLP:journals/corr/EshelCRMYL17}, à des données textuelles bruitées. Un des remèdes consiste à corriger les données délivrées par la ROC \cite{DBLP:conf/taln/SagotG14}, idéalement de manière automatique, lesquelles seront ensuite exploitées dans les différentes tâches du TAL. Or, si certaines interférences des dispositifs de ROC sont systématiques \cite{stanislawek-2019}, lorsqu’elles sont singulières, cet exercice devient difficile à réaliser. En outre, ainsi que le soulignent \citeasnoun{huynh}, la correction peut, elle aussi, produire des erreurs. 
Les erreurs de la ROC peuvent être regroupées en deux catégories \cite{oger} : celle des erreurs lexicales (\textit{non-word errors}) qui ne représentent pas des mots valides de la langue, par exemple si le mot \og{}Morlincourt\fg{}\footnote{Toponyme français extrait de \textit{Mon village}, J.\ Adam, 1860.} est orthographié comme \og{}Mlolincourtl\fg{}, et celle, beaucoup plus restreinte, des erreurs grammaticales (\textit{real-word errors}) \cite{wisniewski} auxquelles pourraient s'ajouter les erreurs sémantiques (\textit{semantic/context-sensitive errors}), quand p. ex. \og{}Gélons\fg{}\footnote{Peuples de Sarmatie, voisins du Borysthène, dans le contexte  \og{}Tel sur les monts glacés des farouches Gelons\fg{}, \textit{Œuvres de Boileau}, T.\ 2, Boileau, 1836.} devient \og{}Gelons\fg{}, grammaticalement correct mais incorrect dans le contexte donné. Les erreurs liées à la correction automatique sont principalement des erreurs sémantiques \cite{azmi}, p. ex., \og{}M.\ Eyssette\fg{}\footnote{Nom d’un personnage du roman \textit{Le petit chose}, de A.\ Daudet, 1868, corrigé automatiquement avec l'outil JamSpell.} devient \og{}M.\ Cassette\fg{}. 

Si le domaine de la correction automatique de texte est très actif et remonte à plusieurs décennies, depuis les travaux de \citeasnoun{damerau} jusqu’à nos jours \cite{nguyen2021}, il n’existe pas de classification unanime pour une approche standard de correction des textes bruités \cite{DBLP:journals/corr/abs-1203-5255,dumasmilneedwards:tel-01562039,Nguyen-2020}. %\cite{DBLP:journals/csur/NguyenJCD21}). 
Néanmoins, trois grandes méthodes se démarquent : les méthodes exploitant des lexiques, les méthodes sur des modèles de langue statistiques, et les méthodes à base d’apprentissage automatique \cite{petkovic2022impact}. 

Une des questions qui préoccupe actuellement la communauté TAL concerne l’évaluation de l’incidence des erreurs de la ROC sur la REN \cite{chiron:hal-03025508,hamdi:hal-03026931,DBLP:journals/corr/abs-2302-10204}. % et l'influence de ce bruit sur les usages consécutifs \cite{vanStrien-2020} de ces données.
La REN, et particulièrement l’identification des entités nommées (EN) de lieux \cite{vanStrien-2020}, est un moyen efficace pour améliorer l’accès aux informations contenues dans de vastes corpus.
D'ailleurs, \citeasnoun{chiron:hal-03025508} ont montré qu'un nombre important de requêtes d'utilisateurs de la plateforme Gallica\footnote{\url{https://gallica.bnf.fr}} étaient affectées par des termes mal transcrits et non répertoriés dans les dictionnaires habituels. 
Les erreurs de la ROC impactent également d'autres tâches (segmentation de phrases, analyse de dépendances, modélisation de sujets et réglage fin du modèle de langage neuronal) ; par exemple, la tâche de modélisation des sujets (\textit{topic models}) est impactée par la mauvaise qualité de la ROC, car les modèles produits divergent de ceux corrigés à la main \cite{vanStrien-2020}.
Par ailleurs, \citeasnoun{evershed2014correcting} soulignent l'importance de la correction automatique des erreurs de la ROC dans un corpus de journaux avec le logiciel \texttt{overProof\footnote{\url{http://overproof.projectcomputing.com/}}} : le taux d'erreur de mots réduit de plus de 60 \% a permis de réduire de plus de 50 \% le nombre d'articles manqués lors d'une recherche par mots-clés.
%L'impact négatif de ROC est également souligné dans d'autres tâches, notamment la segmentation de phrases, l'analyse de dépendances, la recherche d'information, la modélisation de sujets et le réglage fin du modèle de langage neuronal \cite{vanStrien-2020}.

Lors de leurs expériences, \citeasnoun{DBLP:conf/gis/Koudoro-Parfait21}\footnote{\url{https://github.com/These-SCAI2023/NER_GEO_COMPAR}} 
ont noté que les systèmes de REN ont une certaine robustesse face à la variabilité dans les données. Certaines EN dont la forme est dite \og{}contaminée\fg{} \cite{hamdi:hal-03615997} ont malgré tout été reconnues par des outils tels que \texttt{spaCy} ou \texttt{stanza}, p. ex. \og{}Mlorlincourtl\fg{} (forme contaminée de \og{}Morlincourt\fg{})  est repéré et correctement labellisé. On peut supposer qu'il n'est pas nécessaire de corriger la totalité des EN et que la correction de seulement certaines EN\footnote{Suppression des césures et remplacement des \og{}s longs\fg{} par des \og{}s\fg{}.} \cite{DBLP:conf/konvens/AlexGKT12} améliore les résultats de la REN. 
%Enfin, l'impact de ROC sur d'autres tâches en aval est étudié dans l'étude de \cite{chiron:hal-03025508}, qui ont montré qu'un nombre important de requêtes d'utilisateurs de la plateforme Gallica était affecté par des termes mal océrisés et non répertoriés dans les dictionnaires habituels. L'impact négatif de ROC est également souligné dans d'autres tâches, comme la modélisation des sujets ou l'analyse des dépendances \cite{vanStrien-2020}. --> paragraphe inséré plus haut

% **** 
% Ici dire grosso modo:
% La section suivante présente les deux corpus utilisés pour l'évaluation de l'impact de l'OCR sur la REN. 

% **** à chercher impact de la taille du jeu de caractère sur la qualité de l'OCR, les diacritique rendent la segmentation problématique ?  cas de l'arabe : \cite{amara2004approche}

