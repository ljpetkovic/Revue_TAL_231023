Face au volume croissant des données issues de la numérisation et de la ROC, des problématiques relatives à la qualité de ces données et à leur exploitabilité scientifique émergent, étant donné 
% émergent des problématiques relatives à la qualité de ces données et à leur exploitabilité scientifique étant données 
les erreurs dans les transcriptions de ROC, parfois appelées bruit. Le bruit désigne toutes les erreurs produites par le système de ROC : l’insertion, la suppression, mais aussi la substitution d’un ou plusieurs caractères par d’autres. Les chercheur·ses sont ainsi confronté·e·s aux difficultés d’appliquer des outils informatiques, généralement entraînés sur des données textuelles correctement orthographiées \cite{DBLP:journals/corr/EshelCRMYL17}, à des données textuelles bruitées. Un des remèdes consiste à corriger les données délivrées par la ROC \cite{DBLP:conf/taln/SagotG14}, idéalement de manière automatique, lesquelles seront ensuite exploitées dans les différentes tâches du TAL. Or, si certaines interférences des dispositifs de ROC sont systématiques \cite{stanislawek-2019}, lorsqu’elles sont singulières, cet exercice devient difficile à réaliser. En outre, ainsi que le souligne \cite{huynh} la correction peut, elle aussi, produire des erreurs. 
Les erreurs de ROC peuvent être regroupées en deux catégories \cite{oger} : celles des erreurs lexicales (angl. \textit{non-word errors}) qui ne représentent pas des mots valides de la langue, p. ex. si le mot “Morlincourt\footnote{Toponyme français extrait, \textit{Mon village}, J.\ Adam, 1860.}” est écrit “Mlolincourtl”, et celles, beaucoup plus rares, des erreurs grammaticales (angl. \textit{real-word errors}) \cite{wisniewski} auxquelles pourraient s'ajouter les erreurs sémantiques (angl. \textit{semantic/context-sensitive errors}), quand p. ex. “Gélons\footnote{Peuples de Sarmatie, voisins du Borysthène, dans le contexte  \og{}Tel sur les monts glacés des farouches Gelons\fg{}, \textit{Œuvres de Boileau}, T.\ 2, Boileau, 1836.}” devient “Gelons”, grammaticalement correct mais incorrect dans le contexte donné. Les erreurs liées à la correction automatique sont principalement des erreurs sémantiques \cite{azmi}, p. ex., “M.\ Eyssette\footnote{Nom d’un personnage du roman \textit{Le petit chose}, de A.\ Daudet, 1868, corrigé automatiquement avec l'outil JamSpell.}” devient “M.\ Cassette”. 

Si le domaine de la correction automatique de texte est très actif et remonte à plusieurs décennies, depuis les travaux de \cite{damerau} jusqu’à nos jours \cite{nguyen2021}, il n’existe pas de classification unanime pour une approche standard de correction des textes bruités (\cite{DBLP:journals/corr/abs-1203-5255}, \cite{dumasmilneedwards:tel-01562039}, \cite{Nguyen-2020}, %\cite{DBLP:journals/csur/NguyenJCD21}). 
Néanmoins, trois grandes méthodes se démarquent: méthodes exploitant des lexiques, méthodes sur des modèles de langue statistiques, et méthodes à base d’apprentissage automatique \cite{petkovic2022impact}. 

La REN et particulièrement l’identification des entités nommées (EN) de lieux \cite{vanStrien-2020} est un moyen efficace pour améliorer l’accès aux informations contenues dans de vastes corpus. Une des questions qui préoccupe actuellement les chercheur·ses en TAL concerne l’évaluation de l’incidence des erreurs de ROC sur la REN (\cite{chiron:hal-03025508}, \cite{hamdi:hal-03026931}, \cite{DBLP:journals/corr/abs-2302-10204}) et l’influence de ce bruit sur les usages consécutifs \cite{vanStrien-2020} de ces données. Lors de leurs expériences \cite{DBLP:conf/gis/Koudoro-Parfait21} %\footnote{\url{https://github.com/These-SCAI2023/NER_GEO_COMPAR}} 
ont noté que les systèmes de REN ont une certaine robustesse face à la variabilité dans les données. Certaines EN dont la forme est dite \og{}contaminée\fg{} \cite{hamdi:hal-03615997} ont malgré tout été reconnues par des outils tels que \texttt{spaCy}\footnote{Ce système de REN contient une stratégie d'intégration de mots utilisant des fonctionnalités de sous-mots et les plongements ``Bloom'', ainsi qu'un réseau neuronal convolutif avec des connexions résiduelles. Le système est conçu pour offrir un bon équilibre entre efficacité, précision et adaptabilité \cite{bhavani}. Cela peut expliquer sa robustesse lors de l'extraction des EN contaminées.} ou \texttt{stanza}, p. ex. “Mlorlincourtl” (forme contaminée de “Morlincourt”)  est repéré et correctement labellisé. On peut supposer qu'il n'est pas nécessaire de corriger la totalité des EN et que la correction de seulement certaines EN\footnote{Suppression des césures et remplacement des ``s longs'' par des ``s''.} \cite{DBLP:conf/konvens/AlexGKT12} améliore les résultats de la REN. Enfin, l'impact de ROC sur d'autres tâches en aval est étudié dans l'étude de \cite{chiron:hal-03025508}, qui ont montré qu'un nombre important de requêtes d'utilisateurs de la plateforme Gallica était affecté par des termes mal océrisés et non répertoriés dans les dictionnaires habituels. L'impact négatif de ROC est également souligné dans d'autres tâches, comme la modélisation des sujets ou l'analyse des dépendances \cite{vanStrien-2020}. 

% **** 
% Ici dire grosso modo:
% La section suivante présente les deux corpus utilisés pour l'évaluation de l'impact de l'OCR sur la REN. 

% **** à chercher impact de la taille du jeu de caractère sur la qualité de l'OCR, les diacritique rendent la segmentation problématique ?  cas de l'arabe : \cite{amara2004approche}

