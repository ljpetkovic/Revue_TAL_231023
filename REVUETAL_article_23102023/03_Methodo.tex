%\subsubsection{Terminologie employée dans cet article }
%En tant que chercheurs au croisement des TAL et des HN, n
%--------------------- redondant ? cf. les remarques de Damien et de Joseph sur la trivialité d'expliquer les termes couramment acceptés dans la communauté TAL

%Nous jugeons pertinent d'expliciter dans cette partie certains termes du lexique que nous employons afin de rendre nos propos plus largement accessibles.

%\begin{description}
%   % \item [Jspll :]  -- modèle pré-entraîné de JamSpell pour une langue donnée, 
%  %  \item [ELTeC] modèle entraîné sur 40\% de la collection ELTeC pour une langue donnée. 
%    \item [Configuration:] Nous nommons configuration l'association de la version (ROC) et de la REN, par exemple les résultats de la configuration Kraken--\texttt{spaCy\_lg}.
%    \item [Contamination:] Nous adoptons le terme contamination proposé par \cite{hamdi:hal-03615997} pour qualifier les entités dont l'orthographe a été modifiée à cause de la transcription fautive de la ROC.
%    \item [Version:] Nous nommons version le texte obtenu par transcription de ROC.
%    \item [Référence:] Nous nommons référence la version orthographiquement correcte des textes.
%    \item [Vérité terrain:] Nous nommons vérité terrain le jeu de données de REN constitué pour l'évaluation. 
%\end{description}

%\subsubsection{Description et constats à propos des versions des outils d'OCR et de NER}

%--------------------OCR (déplacé dans la sous-section 5.1 le 21/02/24)

%Afin de produire différentes transcriptions de ROC, nous avons utilisé deux systèmes de ROC disponibles gratuitement : Kraken \cite{kiessling2019kraken} et Tesseract \cite{smith2007overview}. Kraken est un outil pour lequel il existe différents modèles neuronaux artificiels de ROC,
%%\footnote{Son modèle de segmentation est le réseau d'étiquetage en graines (angl. \textit{seed-labeling network}), qui est un \textit{U-net} (\og{}réseau entièrement convolutionnel\fg{}) modifié \cite{ronneberger2015} sur la base d'un réseau neuronal résiduel (ResNet) à 34 couches \cite{he2016}, pré-entraîné sur ImageNet. Le modèle de transcription fonctionne comme un classificateur de séquences sans segmentation utilisant un réseau neuronal artificiel pour mapper une image d'une ligne de texte (séquence d'entrée), en une séquence de caractères (séquence de sortie). Le réseau en question est un réseau de neurones convolutif et récurrent, entraîné avec la fonction de perte CTC (angl. \textit{Connexionist Temporal Classification} \cite{graves2006connectionist}).})
%dont un modèle pour le français du XVII\ieme{}siècle \cite{gabay:hal-02577236}\footnote{\url{https://github.com/Heresta/OCR17plus}} ainsi qu'un modèle HTR\footnote{Abbr. angl. \textit{Handwritten Text Recognition} ou la reconnaissance d'écriture manuscrite, basée sur l'extraction du texte à partir des blocs textuels (lignes), contrairement à la méthode de ROC qui traite les caractères individuelles \cite{gabayscicos}.} pour le français, Gallicorpora \cite{pinche_2022_7410529}. En revanche, Tesseract, avec ces nombreux modèles\footnote{Celui utilisé dans le cadre de notre travail est le modèle LSTM \texttt{tessdata\_best}, entraîné sur des données Google. Tesseract propose une analyse de la mise en page intégrée à travers la segmentation des cadres (angl. \textit{box segmentation}), ce qui rend cependant le traitement des mises en page complexes plus difficile \cite{reul2017case}.}, est un outil qui peut également être paramétré pour des tâches particulières, mais qui présente toujours de bonnes performances dans sa configuration par défaut \cite{clausner2020efficient}. Trois modèles de langue neuronaux pour la ROC ont été utilisés dans le cadre des expériences : (i) le modèle de base de Kraken qui permet d'opérer la segmentation\footnote{Le modèle de segmentation est constitué d'un réseau d'étiquetage en graines (angl. \textit{seed-labeling network}), qui est un \textit{U-net} (\og{}réseau entièrement convolutionnel\fg{}) modifié \cite{ronneberger2015} sur la base d'un réseau neuronal résiduel (ResNet) à 34 couches \cite{he2016}, pré-entraîné sur ImageNet.} et la transcription\footnote{Le modèle de transcription fonctionne comme un classificateur de séquences sans segmentation qui utilise un réseau neuronal artificiel pour mapper une image d'une ligne de texte (séquence d'entrée), en une séquence de caractères (séquence de sortie). Le réseau de neurones convolutif et récurrent a été entraîné avec la fonction de perte CTC (angl. \textit{Connexionist Temporal Classification} \cite{graves2006connectionist}).} d'un PDF, (ii) le modèle par défaut de Tesseract\footnote{Le modèle par défaut de Tesseract est l'anglais. Nous précisons la langue Tess. eng, Tess. fr et Tess. suivant les bonnes pratiques de \#RègledeBender \cite{bender-friedman-2018-data}.} et (iii) son modèle entraîné sur du français ou du portugais contemporain\footnote{\url{https://doc.ubuntu-fr.org/tesseract-ocr}}.
%% Pour chaque roman, nous aurons donc le texte de référence (voir les Tableaux  \ref{tab:TGBFra}, \ref{tab:ELTeCFra}, \ref{tab:ELTeCEng} et \ref{tab:ELTeCPor} pour les détails) et trois versions de ROC différentes (i) Kraken, (ii) Tesseract par défaut (Tess. en) et (iii) la version transcrite avec les modèles de Tesseract pour chacune des langues des corpus (français (Tess fr) et portugais (Tess pt)).
 
%%--------------------Correction automatique (déplacé dans la sous-section 6.1 le 21/02/24)
%
%Concernant la correction automatique des textes issus des transcriptions ROC, nous avons utilisé JamSpell, un outil développé en Python qui exploite le modèle de langue trigramme statistique\footnote{Pour plus de détails sur ce modèle cf. \url{https://habr.com/en/articles/346618/}.} (grain mot) et dont une partie des fonctionnalités est mise à disposition gratuitement sur le web.
%% \footnote{Malgré de nombreuses tentatives d'installation sur nos machines en local, nous avons dû nous résoudre à le faire tourner dans un environnement Google Colab.}
%De plus, si les modèles de langues pour le français et l'anglais sont accessibles gratuitement (utilisés dans le cadre de cette étude), le modèle portugais est disponible uniquement dans la version payante de JamSpell (cette option n'a pas été testée).
%Nous avons entrainé un modèle de langue pour JamSpell pour chacune des trois langue analysées dans notre étude. Nous avons sélectionné 40\% de chacun des corpus mis à disposition par ELTeC\footnote{soit 4 millions de tokens pour chaque modèle}, nous en avons exclu les textes utilisés pour notre étude.

%--------------------REN (déplacé dans la sous-section 5.1 le 21/02/24)

%\textbf{Pour effectuer la tâche de REN nous avons utilisés les chaînes de traitements des boîtes à outils pour le TAL \texttt{spaCy}\footnote{version 3.5.1} et \texttt{stanza}\footnote{1.5.0}. 
%\texttt{SpaCy} est un système qui contient une stratégie d'intégration de mots utilisant des fonctionnalités de sous-mots et les plongements ``Bloom'' (angl. \textit{Bloom embeddings})\footnote{structures de données probabilistes qui permet de réduire la dimension des vecteurs \url{https://explosion.ai/blog/bloom-embeddings}}, ainsi qu'un réseau neuronal convolutif avec des connexions résiduelles. Cela peut expliquer sa robustesse lors de l'extraction des EN contaminées. \texttt{spaCy} propose des modèles de langue pour le français\footnote{\url{https://spacy.io/models/fr}, \texttt{fr\_core\_news\_lg}}, le portugais\footnote{\url{https://spacy.io/models/pt}, \texttt{pt\_core\_news\_lg}} et l'anglais\footnote{\url{https://spacy.io/models/en}, \texttt{en\_core\_web\_lg}} (\textit{large}). Nous avons favorisé l'usage du modèle ``large''  \texttt{spaCy\_lg} plutôt que ``small'' car la différence principale entre les deux modèles pour les trois langues tient à l'ajout de la vectorisation et des plongements de mots (angl. \textit{embeddings}) dans l'entraînement du modèle \textit{large}\footnote{Modèle anglais : Explosion Vectors (OSCAR 2109 + Wikipedia + OpenSubtitles + WMT News Crawl) (Explosion). Modèles français et portugais :
%Explosion fastText Vectors (CBOW, OSCAR Common Crawl + Wikipedia) (Explosion).}.}
%
%
%\textbf{\texttt{Stanza} se présente comme la suite d'outils état de l'art pour le TAL, s'appuyant sur des réseaux neuronaux et des modules construits avec la bibliothèque PyTorch. pour la tâche de REN \texttt{Stanza}\footnote{\url{https://stanfordnlp.github.io/stanza/ner_models.html}} propose un modèle de langue français\footnote{Modèle entraîné sur les données de WikiNER.} et un pour l'anglais\footnote{Modèle entraîné sur les données de \textsc{CoNLL03}}, il n'existe pas de modèle pour le portugais. 
%% TODO : enrichier la description de stanza ?
%Les performances pour le portugais sont évaluées sur les configurations comprenant \texttt{spaCy\_lg} uniquement.}
%
%Enfin, il est intéressant de constater que \texttt{spaCy} et \texttt{stanza} utilisent des jeux d'étiquettes différents pour l'annotation des lieux selon les langues : le français et le portugais utilisent l'étiquette \texttt{LOC} et l'anglais utilise les étiquettes \texttt{LOC} et \texttt{GPE} ou \og{}entité géopolitique\fg{}. 
%
%Nous avons donc procédé aux évaluations des différentes configurations présentées dans le tableau \ref{tab:config}.
%
%\begin{table}[h!]
%    \centering
%   %\scriptsize{\begin{tabular}{|c|cc|cc|cc|cc|}
%
%\hline   
%ROC &  \multicolumn{2}{l|}{Kraken} &  \multicolumn{2}{l|}{Tess. en} & \multicolumn{2}{l|}{Tess. fr} &  \multicolumn{2}{l|}{Tess. pt} \\ 
%\hline
%REN & \textit{sp}} & \textit{st} & \textit{sp}} & \textit{st}&\textit{sp}} & \textit{st} &\textit{sp}} & \textit{st} \\
%\hline
%non corr. &\ding{51}} & \ding{51} & \ding{51}} & \ding{51}&\ding{51}} & \ding{51} &\ding{51}} & \ding{53} \\
%Jspll-pretr. &\ding{51}} & \ding{51} & \ding{51}} & \ding{51}&\ding{51}} & \ding{51} &\ding{53}} & \ding{53} \\
%Jspll-ELTeC  &\ding{51}} & \ding{51} & \ding{51}} & \ding{51}&\ding{51}} & \ding{51} &\ding{51}} &\ding{53}  \\
%  \hline
%\end{tabular}}

% \begin{tabular}{|c|cccc|}
% \hline
% ROC          & \multicolumn{1}{l|}{Kraken}                 & \multicolumn{1}{l|}{Tess. en}               & \multicolumn{1}{l|}{Tess. fr}               & \multicolumn{1}{l|}{Tess. pt} \\ \hline
% REN          & \multicolumn{4}{c|}{\textbackslash{}textttspaCy\_lg}                                                                                                                    \\ \hline
% non corr.    & \multicolumn{1}{c|}{\textbackslash{}ding51} & \multicolumn{1}{c|}{\textbackslash{}ding51} & \multicolumn{1}{c|}{\textbackslash{}ding51} & \textbackslash{}ding51        \\
% Jspll-pretr. & \multicolumn{1}{c|}{\textbackslash{}ding51} & \multicolumn{1}{c|}{\textbackslash{}\ding51 & \multicolumn{1}{c|}{\textbackslash{}ding51} & \textbackslash{}ding53        \\
% Jspll-ELTeC  & \multicolumn{1}{c|}{\textbackslash{}ding51} & \multicolumn{1}{c|}{\textbackslash{}ding51} & \multicolumn{1}{c|}{\textbackslash{}ding51} & \textbackslash{}ding51        \\ \hline
% \end{tabular}



% \scriptsize{\begin{tabular}{|c|c|c|c|c|}

% \hline   
% ROC &  Kraken &  Tess. en & Tess. fr &  Tess. pt \\ 
% \hline
% REN & \multicolumn{2}{c|}{\texttt{spaCy\_lg}} &  &  & \\
% % \hline
% non corr. &\ding{51} & \ding{51} &\ding{51} &\ding{51} \\
% Jspll-pretr. &\ding{51} & \ding{51} &\ding{51}&\ding{53} \\
% Jspll-ELTeC  &\ding{51} & \ding{51} &\ding{51}  &\ding{51}  \\
%   \hline
% \end{tabular}}

\scriptsize{
\begin{tabular}{|c|cccc|}
\hline
ROC          & \multicolumn{1}{l|}{Kraken}                 & \multicolumn{1}{l|}{Tess.en}               & \multicolumn{1}{l|}{Tess.fr}               & \multicolumn{1}{l|}{Tess.pt} \\ \hline
REN          & \multicolumn{4}{c|}{\texttt{spaCy\_lg}}                                                                                                                    \\ \hline
non corr.    & \multicolumn{1}{c|}{\ding{51}} & \multicolumn{1}{c|}{\ding{51}} & \multicolumn{1}{c|}{\ding{51}} & \ding{51}        \\
Jspll-pretr. & \multicolumn{1}{c|}{\ding{51}} & \multicolumn{1}{c|}{\ding{51}} & \multicolumn{1}{c|}{\ding{51}} & \ding{53}        \\
Jspll-ELTeC  & \multicolumn{1}{c|}{\ding{51}} & \multicolumn{1}{c|}{\ding{51}} & \multicolumn{1}{c|}{\ding{51}} & \ding{51}        \\ \hline
\end{tabular}
}

%    \caption{Ensemble des configurations que nous évaluons dans cette étude. \texttt{spaCy\_lg}: sp ; \texttt{stanza} : st.}
%    \label{tab:config}
%\end{table}
% % \item [Jspll :]  -- modèle pré-entraîné de JamSpell pour une langue donnée, 
  %  \item [ELTeC] modèle entraîné sur 40\% de la collection ELTeC pour une langue donnée. 


% différence d'étiquetage \texttt{GPE} vs. \texttt{LOC.} pour le spaCy anglais

% Kraken\footnote{\url{https://github.com/mittagessen/kraken}} et Tesseract \cite{smith2007overview}\footnote{\url{https://github.com/tesseract-ocr/tesseract}}, et à ces transcriptions OCR corrigées avec l’outil \texttt{JamSpell}\footnote{\url{https://github.com/bakwc/JamSpell}}.
% Stanza
% versions d'OCR
% Nerval

