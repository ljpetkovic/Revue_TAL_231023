Dans ce travail, nous avons mené des expériences sur la correction automatique des contaminations de la ROC, avec l'objectif de mesurer l'impact de ces corrections sur la reconnaissance d'EN spatiales. Notre étude s'appuie sur les corpus littéraires ELTeC (ouvrages en anglais, français et portugais), ainsi que sur celui de la TGB (ouvrages en français), dont les versions de ROC ont été corrigées à l'aide de deux modèles de l'outil JamSpell (l'un fournit par défaut pour l'anglais et le français et l'autre entraîné sur le corpus ELTeC selon la langue adéquate). Les résultats ont montré que, contre-intuitivement, la correction automatique introduit des biais (notamment les sur-corrections) dans les données textuelles et que le gain apporté par les corrections justes n'était pas considérable. En revanche, pour les ouvrages en anglais et en français transcrits avec Tesseract, nous avons constaté un gain de performance léger mais quasi systématique du correcteur JamSpell--ELTeC. A contrario le modèle pré-entraîné semble plus souvent apporter des sur-corrections. Dans la suite de notre travail, nous nous appuierons sur l'utilisation d'un autre outil qui utilise des réseaux de neurones, qui serait susceptible de corriger automatiquement les contaminations de la ROC de manière plus probante.