%Dans cette partie, nous présentons les corpus utilisés pour l'évaluation de l'impact de la correction automatique des OCR sur les outils de REN.
Pour chaque corpus ELTeC français, anglais et portugais, nous avons à notre disposition le texte de référence, généralement de très bonne qualité, de chaque roman (voir les tableaux \ref{tab:ELTeCFra}, \ref{tab:ELTeCEng} et \ref{tab:ELTeCPor}. Concernant les texte de la Très Grande Bibliothèque (tableau \ref{tab:TGBFra}), les textes de références dont nous disposons sont très fautifs. \textbf{Pour chaque texte nous disposons donc de deux versions de ROC différentes (i) Kraken\footnote{\url{https://kraken.re/2.0.0/index.html}} \cite{kiessling2019escriptorium}, (ii) la version transcrite de Tesseract pour chacune des langues des corpus (anglais (Tess. en), français (Tess fr) et portugais (Tess pt)). }

\subsection{La Très Grande Bibliothèque (TGB)}
 La TGB est une bibliothèque de 128 441 documents français en mode texte %(ROC non corrigée)
, issus des collections Gallica de la BnF. Le corpus provient majoritairement de l’édition du XIX\ieme{}siècle et couvre différentes thématiques (littérature, histoire, droit, philosophie, etc.). Les imprimés de cette période sont libres de droits (contrairement à ceux du XX\ieme{}s.). La TGB compte 128 441 textes soit 44,16 Go écrits par 58 287 auteurs et disponibles au format XML-TEI. Les documents sont pour la grande majorité, 95 479 œuvres\footnote{Un nombre considérable de documents représentant des rééditions de textes plus anciens.}, datées du XIX\ieme{}s., 7 294 œuvres sont datées du XVIII\ieme{}s., 54 du XX\ieme{}s. et 24 du XVII\ieme{}s.
%\begin{itemize}
  %      \item XVII\ieme{} : 24
   %     \item XVIII\ieme{} : 7294
    %    \item XIX\ieme{} : 95479
     %   \item XX\ieme{} : 54
    %\end{itemize}

Tous les documents sont liés à leur notice dans le catalogue général de la BnF et pointent vers leur instance Gallica. Les auteurs sont liés à leur notice de personne du catalogue général. Les textes sont issus d’un traitement de ROC brut sans corrections. Les services de la BnF nous ont précisés que plusieurs moteurs ROC ont été utilisés, Abby étant le principal depuis 2009 et celui utilisé en interne. Néanmoins certains prestataires utilisaient des solutions internes, ou un mix de ROC et certains marchés incluaient une phase de correction humaine post-ROC.
%les réglages ont beaucoup d'impact pour un même outil
Un score de confiance est fournie par les services de la BnF sans précision sur la métrique utilisée. Certains textes ont cependant un taux de confiance avancé de ROC de 100\%. Ainsi, 127 267 documents de la TGB (99\% du corpus) ont été indexés par les catalogueurs de la BnF, selon la classification Dewey, à partir de laquelle dix sous-corpus thématiques (inégalement représentés) ont été créés. %(Tableau \ref{tab:TGB})
On compte 35 710 oeuvres pour la catégorie Littérature (Belles-lettres) et 1 861 pour la catégorie Langues romanes, Français. Nous avons extrait une dizaine d'oeuvre de ces deux catégories pour constituer notre corpus de PDF comportant des traits permettant d'illustrer les difficultés de l'application de la ROC à des textes anciens, par exemple la transcription de certaines décorations ou de caractères en capitales stylisées. Les informations générales et le nombre d'EN de lieux reconnues dans ces textes de référence par les deux outils \texttt{spaCy} et \texttt{stanza} sont présentées dans le tableau \ref{tab:TGBFra}. 

%\begin{table}
%\small
 %   \centering
  %  \begin{tabular}{|p{5cm}|p{3cm}|}%|p{1cm}}

\hline
\small{\textbf{Thématique}} & \small{\textbf{\# Documents}} \\ %& \small{} & \small{}\\
\hline
  \small{Littérature (Belles-lettres)} & \multicolumn{1}{r|}{35710}  \\
  \small{Histoire de la France (depuis 486)} & \multicolumn{1}{r|}{28885}\\
  \small{Droit} & \multicolumn{1}{r|}{23776}\\
  \small{Économie domestique. Vie à la maison} & \multicolumn{1}{r|}{19622}\\
  \small{Les arts} & \multicolumn{1}{r|}{5653}\\
  \small{Astronomie et sciences connexes} & \multicolumn{1}{r|}{4307}\\
  \small{Journalisme, édition. Journaux} & \multicolumn{1}{r|}{3824}\\
  \small{Religion} & \multicolumn{1}{r|}{2576}\\
  \small{Langues romanes. Français} & \multicolumn{1}{r|}{1861}\\
  \small{Philosophie et disciplines connexes} & \multicolumn{1}{r|}{1491}\\
  \hline
\end{tabular}



   % \caption{Répartition thématique du Corpus TGB. 
   %\label{tab:TGB}}
%\end{table}

\begin{table}[h!]
    \centering
    \small
    % ordre croissant (nb de mots dans le texte)
 \resizebox{\textwidth}{!}{
 \begin{tabular}{|p{3.6cm}|p{2.3cm}|p{.8cm}|p{.9cm}|p{1cm}|p{1.2cm}|}

\hline
\scriptsize{\textbf{Ouvrage}} & \scriptsize{\textbf{Auteur}} & \scriptsize{\textbf{Année}}  & \scriptsize{\textbf{Pages}}& \scriptsize{\textbf{Tokens}} & \scriptsize{\textbf{\texttt{spaCy\_lg}}}  \\
\hline
\scriptsize{\textit{L'Alsace et la Lorraine}} & L.\ Longret& \multicolumn{1}{r|}{1873} &\multicolumn{1}{r|}{2}  &\multicolumn{1}{r|}{357}&\multicolumn{1}{r|}{13}\\
  \hline
    \scriptsize{\textit{La Grèce libre}} &  A.\ Bignan & \multicolumn{1}{r|}{1821} &\multicolumn{1}{r|}{20} &\multicolumn{1}{r|}{1 027}&\multicolumn{1}{r|}{35} \\
  \hline
  \scriptsize{\textit{Poësies diverses}} & Inconnu & \multicolumn{1}{r|}{1745} & \multicolumn{1}{r|}{10} &\multicolumn{1}{r|}{1 502}&\multicolumn{1}{r|}{32}\\
  \hline
    \scriptsize{\textit{Les dernières Étrivières [...] }} &B.\ Bonafoux &\multicolumn{1}{r|}{1877}  &\multicolumn{1}{r|}{22} &\multicolumn{1}{r|}{2 320}&\multicolumn{1}{r|}{29} \\
  \hline
    \scriptsize{\textit{M.\ de L'Espinasse [...] }} & D.\ L.\ Baric & \multicolumn{1}{r|}{1851} &\multicolumn{1}{r|}{20}  &\multicolumn{1}{r|}{3 058}&\multicolumn{1}{r|}{102}\\
  \hline
    \scriptsize{\textit{Adélaïde de Mariendal, drame en cinq actes}} & Inconnu & \multicolumn{1}{r|}{1783} &\multicolumn{1}{r|}{100}  &\multicolumn{1}{r|}{15 344}&\multicolumn{1}{r|}{276}\\
  \hline
  \scriptsize{\textit{Œuvres du seigneur de Brantome. Tome 14 }} & P.\ de  \mbox{Bourdeille} Sgr de  \mbox{Brantôme}  & \multicolumn{1}{r|}{1779} & \multicolumn{1}{r|}{255} &\multicolumn{1}{r|}{49 084}&\multicolumn{1}{r|}{844}\\
  \hline
\scriptsize{\textit{Souvenirs d'un vieux mélomane}} &  A.\ Pontmartin &\multicolumn{1}{r|}{1879}  & \multicolumn{1}{r|}{350} &\multicolumn{1}{r|}{61 872}&\multicolumn{1}{r|}{659}\\
  \hline
  \scriptsize{\textit{La lyre des petits enfants}} &A.\ Cordier & \multicolumn{1}{r|}{1857} &\multicolumn{1}{r|}{357} &\multicolumn{1}{r|}{62 639}&\multicolumn{1}{r|}{646} \\
  \hline


    


  

 

\end{tabular}
}
%  \begin{tabular}{|p{3.6cm}|p{2.3cm}|p{.8cm}|p{.9cm}|p{1cm}|p{1.2cm}|p{.9cm}|}

% \hline
% \small{\textbf{Ouvrage}} & \small{\textbf{Auteur}} & \small{\textbf{Année}}  & \small{\textbf{Pages}}& \small{\textbf{Mots}} & \small{\textbf{\texttt{spaCy\_lg$^*$}}}& \small{\texttt{stanza}} \\
% \hline
%   \small{\textit{La lyre des petits enfants}} &A.\ Cordier & \multicolumn{1}{r|}{1857} &\multicolumn{1}{r|}{357} &\multicolumn{1}{r|}{62639}&\multicolumn{1}{r|}{646} &\multicolumn{1}{r|}{447}\\
%   \hline
%   \small{\textit{La Grèce libre}} &  A.\ Bignan & \multicolumn{1}{r|}{1821} &\multicolumn{1}{r|}{20} &\multicolumn{1}{r|}{1027}&\multicolumn{1}{r|}{35} &\multicolumn{1}{r|}{19}\\
%   \hline
%   \small{\textit{Les dernières Étrivières [...] }} &B.\ Bonafoux &\multicolumn{1}{r|}{1877}  &\multicolumn{1}{r|}{22} &\multicolumn{1}{r|}{2320}&\multicolumn{1}{r|}{29} &\multicolumn{1}{r|}{20}\\
%   \hline
%     \small{\textit{Poësies diverses}} & Inconnu & \multicolumn{1}{r|}{1745} & \multicolumn{1}{r|}{10} &\multicolumn{1}{r|}{1502}&\multicolumn{1}{r|}{32}&\multicolumn{1}{r|}{11}\\
%   \hline
%   \small{\textit{M.\ de L'Espinasse [...] }} & D.\ L.\ Baric & \multicolumn{1}{r|}{1851} &\multicolumn{1}{r|}{20}  &\multicolumn{1}{r|}{3058}&\multicolumn{1}{r|}{102}&\multicolumn{1}{r|}{91}\\
%   \hline
%   \small{\textit{Adélaïde de Mariendal, drame en cinq actes}} & Inconnu & \multicolumn{1}{r|}{1783} &\multicolumn{1}{r|}{100}  &\multicolumn{1}{r|}{15344}&\multicolumn{1}{r|}{276}&\multicolumn{1}{r|}{217}\\
%   \hline
%   \small{\textit{Souvenirs d'un vieux mélomane}} &  A.\ Pontmartin &\multicolumn{1}{r|}{1879}  & \multicolumn{1}{r|}{350} &\multicolumn{1}{r|}{61872}&\multicolumn{1}{r|}{659}&\multicolumn{1}{r|}{598}\\
%   \hline
%    \small{\textit{L'Alsace et la Lorraine}} & L.\ Longret& \multicolumn{1}{r|}{1873} &\multicolumn{1}{r|}{2}  &\multicolumn{1}{r|}{357}&\multicolumn{1}{r|}{13}&\multicolumn{1}{r|}{12}\\
%   \hline
%    \small{\textit{Œuvres du seigneur de Brantome. Tome 14 }} & P.\ de  \mbox{Bourdeille} Sgr de  \mbox{Brantôme}  & \multicolumn{1}{r|}{1779} & \multicolumn{1}{r|}{255} &\multicolumn{1}{r|}{49084}&\multicolumn{1}{r|}{844}&\multicolumn{1}{r|}{507}\\
%   \hline
 

% \end{tabular}

    \caption{Statistiques sur le sous-corpus TGB. (spaCy\_lg : modèle large de \texttt{spaCy}. Les deux dernières colonnes indiquent le nombre des EN).}
    \label{tab:TGBFra}
\end{table}



\subsection{European Literary Text Collection (ELTeC)}

Entre 2017 et 2022 l'action COST \textit{Distant Reading for European Literary History} (CA16204) a constitué une collection de corpus de textes littéraires dans plusieurs langues européennes dont certains ont pour source le site web du projet Gutenberg, Gallica mais aussi la Bibliothèque électronique du Québec\footnote{\url{http://beq.ebooksgratuits.com/}} développée par Jean-Yves Dupuis entre 1998 et 2018. \textbf{Les textes disponibles sont tous de bonnes qualités. L'action COST a mis en place une liste de critère\footnote{https://github.com/distantreading/WG1/wiki/E5C-discussion-paper} permettant la sélection des oeuvres entrant dans le périmètre d'une collection ELTeC. La question de la qualité intrinsèque du texte n'étant pas clairement mise en question on peu en conclure qu'implicitement il est attendu que les textes romanesques soient de la qualité d'une édition scientifique donc exempte de fautes d'ORC.} Le but de l'action est de rendre disponible des œuvres romanesques pour la conception, l'évaluation et l'utilisation d'outils et de méthodes d'analyse multilingues des textes littéraires. ELTeC compile des corpus de romans pour plus d'une vingtaine de langues européennes. Les collections française, anglaise et portugaise comprennent chacune une centaine de romans océrisés publiés entre le milieu du XIX\ieme{}siècle et le début du XX\ieme{}siècle. Les romans sont disponibles en plusieurs formats : le format texte brut (.txt), un encodage TEI et un encodage TEI enrichi par une annotation morphosyntaxique.
Pour cette étude, nous avons travaillé avec des textes collectés dans les collections française (11), anglaise (9) et portugaise (4). Le corpus portugais est d'une taille restreinte du fait de difficultés à rassembler des textes de référence et des PDFs d'une qualité équivalente à ceux des corpus français et anglais. Les tableaux \ref{tab:ELTeCFra}, \ref{tab:ELTeCEng} et \ref{tab:ELTeCPor} présentent les informations générales sur les corpus conçus à partir des collections ELTeC. Ils comprennent le nombre d'EN de lieux reconnues dans chacun d'eux par les outils \texttt{spaCy} et \texttt{stanza}.

%\begin{table}
 %   \centering
  %  \begin{tabular}{|p{2cm}|p{3cm}|p{1.2cm}|}%|p{1cm}}

\hline
\small{\textbf{Langue}} & \small{\textbf{Documents}} & \small{\textbf{Mots}} \\ %& \small{} & \small{}\\
\hline
  \small{tchèque} & \multicolumn{1}{r|}{\small{100}} & \multicolumn{1}{r|}{\small{5621667}} \\
  \small{allemand} & \multicolumn{1}{r|}{\small{100}} & \multicolumn{1}{r|}{\small{12738842}}\\
  \small{anglais} & \multicolumn{1}{r|}{\small{100}} & \multicolumn{1}{r|}{\small{12227703}}\\
  \small{français} & \multicolumn{1}{r|}{\small{100}} & \multicolumn{1}{r|}{\small{8712219}}\\
  \small{suisse allemand} & \multicolumn{1}{r|}{\small{100}} & \multicolumn{1}{r|}{\small{6408326}}\\
  \small{hongrois} & \multicolumn{1}{r|}{\small{100}} & \multicolumn{1}{r|}{\small{6948590}}\\
  \small{polonais} & \multicolumn{1}{r|}{\small{100}} & \multicolumn{1}{r|}{\small{8500172}}\\
  \small{portugais} & \multicolumn{1}{r|}{\small{100}} & \multicolumn{1}{r|}{\small{6799385}}\\
  \small{roumain} & \multicolumn{1}{r|}{\small{100}} & \multicolumn{1}{r|}{\small{5951910}}\\
  \small{slovène} & \multicolumn{1}{r|}{\small{100}} & \multicolumn{1}{r|}{\small{5682120}}\\
  \hline
\end{tabular}



   % \caption{Répartition par langue du Corpus ELTeC.\label{tab:ELTeC_nb_docs}}
%\end{table} 

\begin{table}[h!]
    \small
    \begin{subtable}[h]{1\textwidth}\resizebox{\textwidth}{!}{   
    %  \begin{tabular}%{|p{3.6cm}|l|p{.9cm}|p{1cm}|p{1cm}|p{1.2cm}|p{1cm}|}
% {|p{3.6cm}|p{2.3cm}|p{.8cm}|p{.9cm}|p{1cm}|p{1.2cm}|p{.9cm}|}
% \hline
% \small{\textbf{Ouvrage}} & \small{\textbf{Auteur}} & \small{\textbf{Année}}  & \small{\textbf{Pages}}& \small{\textbf{Mots}} & \small{\textbf{\texttt{spaCy\_lg}}}& \small{\texttt{stanza}} \\
% \hline
% \small{\textit{Mon village}}& J.\ Adam & \multicolumn{1}{r|}{1860} & \multicolumn{1}{r|}{200}  &\multicolumn{1}{r|}{20938}&\multicolumn{1}{r|}{213}&\multicolumn{1}{r|}{152}\\
%  \hline
%    \small{\textit{La Belle rivière} }& G.\ Aimard & \multicolumn{1}{r|}{1894} & \multicolumn{1}{r|}{339} &\multicolumn{1}{r|}{137392}&\multicolumn{1}{r|}{1004}&\multicolumn{1}{r|}{959}\\
% \hline 
% \small{\textit{Les trappeurs de l'Arkansas}}& G.\ Aimard & \multicolumn{1}{r|}{1858} & \multicolumn{1}{r|}{450}  &\multicolumn{1}{r|}{91119}&\multicolumn{1}{r|}{646}&\multicolumn{1}{r|}{606}\\
% \hline
% \small{\textit{Marie-Claire}}& M.\ Audoux  & \multicolumn{1}{r|}{1925} & \multicolumn{1}{r|}{120}   &\multicolumn{1}{r|}{35780}&\multicolumn{1}{r|}{101}&\multicolumn{1}{r|}{108}\\
%   \hline
%   \small{\textit{Albert Savarus. Une fille d'Ève}}& H.\ de Balzac & \multicolumn{1}{r|}{1853} & \multicolumn{1}{r|}{60} &\multicolumn{1}{r|}{79924}&\multicolumn{1}{r|}{682}&\multicolumn{1}{r|}{684}\\
%   \hline
%   \small{\textit{La petite Jeanne}} & Z.\ Carraud & \multicolumn{1}{r|}{1884} & \multicolumn{1}{r|}{220} &\multicolumn{1}{r|}{53212}&\multicolumn{1}{r|}{316}&\multicolumn{1}{r|}{95}\\
%   \hline
%     \small{\textit{Le château de Pinon, vol.\ I}} & \small{G.\ A.\ Dash} & \multicolumn{1}{r|}{1844} & \multicolumn{1}{r|}{332} &\multicolumn{1}{r|}{44246}&\multicolumn{1}{r|}{271}&\multicolumn{1}{r|}{311}\\
%   \hline
%   \small{\textit{Le petit chose}} & A.\ Daudet & \multicolumn{1}{r|}{1868} & \multicolumn{1}{r|}{292}  &\multicolumn{1}{r|}{86482}&\multicolumn{1}{r|}{744}&\multicolumn{1}{r|}{580}\\
%   \hline
%    \small{\textit{L'Éducation sentimentale}} & G.\ Flaubert & \multicolumn{1}{r|}{1880} & \multicolumn{1}{r|}{520} &\multicolumn{1}{r|}{150494}&\multicolumn{1}{r|}{1304}&\multicolumn{1}{r|}{1098}\\
%    \hline
%   \small{\textit{Une vie}}& G.\ de Maupassant & \multicolumn{1}{r|}{1883} & \multicolumn{1}{r|}{337} &\multicolumn{1}{r|}{75745}&\multicolumn{1}{r|}{302}&\multicolumn{1}{r|}{312}\\
%   \hline
%  \small{\textit{La nouvelle espérance}}& A.\ de Noailles & \multicolumn{1}{r|}{1903} & \multicolumn{1}{r|}{325}& \multicolumn{1}{r|}{54272}&\multicolumn{1}{r|}{182}&\multicolumn{1}{r|}{236}\\ 
%  \hline
  
% \end{tabular}
\resizebox{\textwidth}{!}{
% ordre par taille (croissant)
 \begin{tabular}%{|p{3.6cm}|l|p{.9cm}|p{1cm}|p{1cm}|p{1.2cm}|p{1cm}|}
{|p{3.6cm}|p{2.3cm}|p{.8cm}|p{.9cm}|p{1cm}|p{1.2cm}|p{.9cm}|}
\hline
\small{\textbf{Ouvrage}} & \small{\textbf{Auteur}} & \small{\textbf{Année}}  & \small{\textbf{Pages}}& \small{\textbf{Tokens}} & \small{\textbf{\texttt{spaCy\_lg}}}& \small{\texttt{stanza}} \\
\hline
\small{\textit{Mon village}}& J.\ Adam & \multicolumn{1}{r|}{1860} & \multicolumn{1}{r|}{200}  &\multicolumn{1}{r|}{20 938}&\multicolumn{1}{r|}{213}&\multicolumn{1}{r|}{152}\\
 \hline
 \small{\textit{Marie-Claire}}& M.\ Audoux  & \multicolumn{1}{r|}{1925} & \multicolumn{1}{r|}{120}   &\multicolumn{1}{r|}{35 780}&\multicolumn{1}{r|}{101}&\multicolumn{1}{r|}{108}\\
  \hline
      \small{\textit{Le château de Pinon, vol.\ I}} & \small{G.\ A.\ Dash} & \multicolumn{1}{r|}{1844} & \multicolumn{1}{r|}{332} &\multicolumn{1}{r|}{44 246}&\multicolumn{1}{r|}{271}&\multicolumn{1}{r|}{311}\\
  \hline
    \small{\textit{La petite Jeanne}} & Z.\ Carraud & \multicolumn{1}{r|}{1884} & \multicolumn{1}{r|}{220} &\multicolumn{1}{r|}{53 212}&\multicolumn{1}{r|}{316}&\multicolumn{1}{r|}{95}\\
  \hline
   \small{\textit{La nouvelle espérance}}& A.\ de Noailles & \multicolumn{1}{r|}{1903} & \multicolumn{1}{r|}{325}& \multicolumn{1}{r|}{54 272}&\multicolumn{1}{r|}{182}&\multicolumn{1}{r|}{236}\\ 
 \hline
   \small{\textit{Une vie}}& G.\ de Maupassant & \multicolumn{1}{r|}{1883} & \multicolumn{1}{r|}{337} &\multicolumn{1}{r|}{75 745}&\multicolumn{1}{r|}{302}&\multicolumn{1}{r|}{312}\\
  \hline
    \small{\textit{Albert Savarus. Une fille d'Ève}}& H.\ de Balzac & \multicolumn{1}{r|}{1853} & \multicolumn{1}{r|}{60} &\multicolumn{1}{r|}{79 924}&\multicolumn{1}{r|}{682}&\multicolumn{1}{r|}{684}\\
  \hline
    \small{\textit{Le petit chose}} & A.\ Daudet & \multicolumn{1}{r|}{1868} & \multicolumn{1}{r|}{292}  &\multicolumn{1}{r|}{86 482}&\multicolumn{1}{r|}{744}&\multicolumn{1}{r|}{580}\\
  \hline
  \small{\textit{Les trappeurs de l'Arkansas}}& G.\ Aimard & \multicolumn{1}{r|}{1858} & \multicolumn{1}{r|}{450}  &\multicolumn{1}{r|}{91 119}&\multicolumn{1}{r|}{646}&\multicolumn{1}{r|}{606}\\
\hline
   \small{\textit{La Belle rivière} }& G.\ Aimard & \multicolumn{1}{r|}{1894} & \multicolumn{1}{r|}{339} &\multicolumn{1}{r|}{137 392}&\multicolumn{1}{r|}{1 004}&\multicolumn{1}{r|}{959}\\
\hline 


   \small{\textit{L'Éducation sentimentale}} & G.\ Flaubert & \multicolumn{1}{r|}{1880} & \multicolumn{1}{r|}{520} &\multicolumn{1}{r|}{150 494}&\multicolumn{1}{r|}{1 304}&\multicolumn{1}{r|}{1 098}\\
   \hline

  
\end{tabular}
    }}
    \caption{Sous-corpus français.\label{tab:ELTeCFra}}
    \end{subtable}

   \begin{subtable}[h]{1\textwidth}
    \small
    %  \begin{tabular}%{|p{3.6cm}|l|p{1cm}|p{1cm}|p{1cm}|p{1.2cm}|p{1cm}|}
% {|p{3.6cm}|p{2.3cm}|p{.8cm}|p{.9cm}|p{1cm}|p{1.2cm}|p{.9cm}|}
% \hline
% \small{\textbf{Ouvrage}} & \small{\textbf{Auteur}} & \small{\textbf{Année}}  & \small{\textbf{Pages}}& \small{\textbf{Mots}} & \small{\textbf{\texttt{spaCy\_lg}}}& \small{\texttt{stanza}} \\
% \hline
%   \small{\textit{Home influence}} & \small{G.\ Aguillar} & \multicolumn{1}{r|}{1847}&\multicolumn{1}{r|}{628} &\multicolumn{1}{r|}{171342}&\multicolumn{1}{r|}{205}&\multicolumn{1}{r|}{244} \\
%   \hline
%   \small{\textit{Auriol}}& W.\ H.\ Ainsworth & \multicolumn{1}{r|}{1844}& \multicolumn{1}{r|}{246} &\multicolumn{1}{r|}{46388}&\multicolumn{1}{r|}{82}&\multicolumn{1}{r|}{55}\\
%   \hline
% \small{\textit{Wuthering Heights}}& E.\ Brontë & \multicolumn{1}{r|}{1847} &  \multicolumn{1}{r|}{764}&\multicolumn{1}{r|}{94986}&\multicolumn{1}{r|}{140}& \multicolumn{1}{r|}{132}\\
% \hline
%  \small{\textit{Coningsby}}& B.\ Disraeli&\multicolumn{1}{r|}{1844} & \multicolumn{1}{r|}{983} &\multicolumn{1}{r|}{101778}&\multicolumn{1}{r|}{634}&\multicolumn{1}{r|}{543}\\
%  \hline
%    \small{\textit{Mary Barton}} &E.\ Gaskell & \multicolumn{1}{r|}{1848} & \multicolumn{1}{r|}{423} &\multicolumn{1}{r|}{161568}&\multicolumn{1}{r|}{290}&\multicolumn{1}{r|}{281}\\
%   \hline  \small{\textit{The Mysteries of London}} &G.\ Reynolds  & \multicolumn{1}{r|}{1844}&\multicolumn{1}{r|}{840} &\multicolumn{1}{r|}{810167}&\multicolumn{1}{r|}{2019}& \multicolumn{1}{r|}{2312}\\
%   \hline  \small{\textit{Modern Flirtations vol.\ 1}} &C.\ Sinclair  & \multicolumn{1}{r|}{1841}&\multicolumn{1}{r|}{386}  &\multicolumn{1}{r|}{189057}&\multicolumn{1}{r|}{502}&\multicolumn{1}{r|}{248}\\
%  % \hline  \small{\textit{The Inheritance of Evil Or, the Consequences of Marrying a Deceased Wife's Sister}} & Felicia Skene &1849 & 186 &\\
%   \hline
%       \small{\textit{Vanity Fair }} &  W.\ M.\ Thackeray & \multicolumn{1}{r|}{1848} &\multicolumn{1}{r|}{624} & \multicolumn{1}{r|}{298568}&\multicolumn{1}{r|}{1492}& \multicolumn{1}{r|}{1164}\\
%   \hline
%     \small{\textit{The Life and Adventures of M.\ Armstrong }} & F.\ Trolloppe & \multicolumn{1}{r|}{1840}&\multicolumn{1}{r|}{387}  & \multicolumn{1}{r|}{189392} &\multicolumn{1}{r|}{187}&\multicolumn{1}{r|}{207}\\
%   \hline
% \end{tabular}

% ordre croissant (nb de mots dans le texte)
\resizebox{\textwidth}{!}{
 \begin{tabular}%{|p{3.6cm}|l|p{1cm}|p{1cm}|p{1cm}|p{1.2cm}|p{1cm}|}
{|p{3.6cm}|p{2.3cm}|p{.8cm}|p{.9cm}|p{1cm}|p{1.2cm}|p{.9cm}|}
\hline
\small{\textbf{Ouvrage}} & \small{\textbf{Auteur}} & \small{\textbf{Année}}  & \small{\textbf{Pages}}& \small{\textbf{Tokens}} & \small{\textbf{\texttt{spaCy\_lg}}}& \small{\texttt{stanza}} \\
\hline
\small{\textit{Auriol}}& W.\ H.\ Ainsworth & \multicolumn{1}{r|}{1844}& \multicolumn{1}{r|}{246} &\multicolumn{1}{r|}{46 388}&\multicolumn{1}{r|}{82}&\multicolumn{1}{r|}{55}\\
  \hline
  \small{\textit{Wuthering Heights}}& E.\ Brontë & \multicolumn{1}{r|}{1847} &  \multicolumn{1}{r|}{764}&\multicolumn{1}{r|}{94 986}&\multicolumn{1}{r|}{140}& \multicolumn{1}{r|}{132}\\
\hline
\small{\textit{Coningsby}}& B.\ Disraeli&\multicolumn{1}{r|}{1844} & \multicolumn{1}{r|}{983} &\multicolumn{1}{r|}{101 778}&\multicolumn{1}{r|}{634}&\multicolumn{1}{r|}{543}\\
 \hline
    \small{\textit{Mary Barton}} &E.\ Gaskell & \multicolumn{1}{r|}{1848} & \multicolumn{1}{r|}{423} &\multicolumn{1}{r|}{161 568}&\multicolumn{1}{r|}{290}&\multicolumn{1}{r|}{281}\\
  \hline 
    \small{\textit{Home influence}} & \small{G.\ Aguillar} & \multicolumn{1}{r|}{1847}&\multicolumn{1}{r|}{628} &\multicolumn{1}{r|}{171 342}&\multicolumn{1}{r|}{205}&\multicolumn{1}{r|}{244} \\
  \hline
  \small{\textit{Modern Flirtations vol.\ 1}} &C.\ Sinclair  & \multicolumn{1}{r|}{1841}&\multicolumn{1}{r|}{386}  &\multicolumn{1}{r|}{189 057}&\multicolumn{1}{r|}{502}&\multicolumn{1}{r|}{248}\\
  \hline
      \small{\textit{The Life and Adventures of M.\ Armstrong }} & F.\ Trolloppe & \multicolumn{1}{r|}{1840}&\multicolumn{1}{r|}{387}  & \multicolumn{1}{r|}{189 392} &\multicolumn{1}{r|}{187}&\multicolumn{1}{r|}{207}\\
  \hline
  \small{\textit{Vanity Fair }} &  W.\ M.\ Thackeray & \multicolumn{1}{r|}{1848} &\multicolumn{1}{r|}{624} & \multicolumn{1}{r|}{298 568}&\multicolumn{1}{r|}{1 492}& \multicolumn{1}{r|}{1 164}\\
  \hline
 \small{\textit{The Mysteries of London}} &G.\ Reynolds  & \multicolumn{1}{r|}{1844}&\multicolumn{1}{r|}{840} &\multicolumn{1}{r|}{810 167}&\multicolumn{1}{r|}{2 019}& \multicolumn{1}{r|}{2 312}\\
 %\hline  \small{\textit{The Inheritance of Evil Or, the Consequences of Marrying a Deceased Wife's Sister}} & Felicia Skene &1849 & 186 &\\
  \hline
      

\end{tabular}
}


    \caption{Sous-corpus anglais.\label{tab:ELTeCEng}}
    \end{subtable}
    
    \begin{subtable}[h]{1\textwidth}
    \small
    %   \begin{tabular}%{|p{3.6cm}|l|p{1cm}|p{1cm}|p{1cm}|p{1.2cm}|p{1cm}|}
% {|p{3.1cm}|p{2.3cm}|p{.8cm}|p{.9cm}|p{1cm}|p{1.2cm}|p{.9cm}|}
% \hline
% \small{\textbf{Ouvrage}} & \small{\textbf{Auteur}} & \small{\textbf{Année}}  & \small{\textbf{Pages}}& \small{\textbf{Mots}} & \small{\textbf{\texttt{spaCy\_lg}}}& \small{\texttt{stanza}} \\
% \hline
%   \small{\textit{Quattro Novelas}} &A.\ Castro Osorio & \multicolumn{1}{r|}{1908} &\multicolumn{1}{r|}{272} &\multicolumn{1}{r|}{50766}&\multicolumn{1}{r|}{353}& \multicolumn{1}{r|}{N/A} \\
%   \hline
%   \small{\textit{Casa de Ramires}} & E.\ de Queiroz &\multicolumn{1}{r|}{1900}  &\multicolumn{1}{r|}{543} &\multicolumn{1}{r|}{107441}&\multicolumn{1}{r|}{3881}&\multicolumn{1}{r|}{N/A} \\
%   \hline
%   \small{\textit{O crime do padre Amoro}} & E.\ de Queiroz& \multicolumn{1}{r|}{1875}&\multicolumn{1}{r|}{620} &\multicolumn{1}{r|}{141700}&\multicolumn{1}{r|}{2362}& \multicolumn{1}{r|}{N/A}\\
%   \hline
%     \small{\textit{Uma familia ingleza}} &J.\ Diniz &  \multicolumn{1}{r|}{1875}&\multicolumn{1}{r|}{360}  &\multicolumn{1}{r|}{122008}&\multicolumn{1}{r|}{994}&\multicolumn{1}{r|}{N/A}\\
%   \hline

% \end{tabular}

% ordre croissant (nb de mots dans le texte)
\resizebox{\textwidth}{!}{
  \begin{tabular}%{|p{3.6cm}|l|p{1cm}|p{1cm}|p{1cm}|p{1.2cm}|p{1cm}|}
{|p{3.1cm}|p{2.3cm}|p{.8cm}|p{.9cm}|p{1cm}|p{1.2cm}|p{.9cm}|}
\hline
\small{\textbf{Ouvrage}} & \small{\textbf{Auteur}} & \small{\textbf{Année}}  & \small{\textbf{Pages}}& \small{\textbf{Tokens}} & \small{\textbf{\texttt{spaCy\_lg}}}& \small{\texttt{stanza}} \\
\hline
  \small{\textit{Quattro Novelas}} &A.\ Castro Osorio & \multicolumn{1}{r|}{1908} &\multicolumn{1}{r|}{272} &\multicolumn{1}{r|}{50 766}&\multicolumn{1}{r|}{353}& \multicolumn{1}{r|}{N/A} \\
  \hline
  \small{\textit{Casa de Ramires}} & E.\ de Queiroz &\multicolumn{1}{r|}{1900}  &\multicolumn{1}{r|}{543} &\multicolumn{1}{r|}{107 441}&\multicolumn{1}{r|}{3 881}&\multicolumn{1}{r|}{N/A} \\
  \hline
      \small{\textit{Uma familia ingleza}} &J.\ Diniz &  \multicolumn{1}{r|}{1875}&\multicolumn{1}{r|}{360}  &\multicolumn{1}{r|}{122 008}&\multicolumn{1}{r|}{994}&\multicolumn{1}{r|}{N/A}\\
  \hline
  \small{\textit{O crime do padre Amoro}} & E.\ de Queiroz& \multicolumn{1}{r|}{1875}&\multicolumn{1}{r|}{620} &\multicolumn{1}{r|}{141 700}&\multicolumn{1}{r|}{2362}& \multicolumn{1}{r|}{N/A}\\
  \hline

\end{tabular}
}
    \caption{Sous-corpus portugais. \label{tab:ELTeCPor}}
    \end{subtable}
    \caption{Statistiques sur les sous-corpus ELTeC.
    % Compte tenu du temps que prend une annotation manuelle, nous n'avons pas annoté tous les corpus et nous ne disposons pas d'une vérité terrain.
    } 
\end{table}

