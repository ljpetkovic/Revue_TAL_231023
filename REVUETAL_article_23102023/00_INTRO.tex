Les techniques de traitement automatique des langues (TAL\footnote{Extraction d'informations et correction automatique de l'orthographe.}) combinées aux méthodes des humanités numériques (HN\footnote{Valorisation des corpus patrimoniaux et reconnaissance optique des caractères (ROC).}) rendent possibles l'exploration et l'exploitation de corpus numérisés à grande échelle.
Depuis les années 2000 les institutions publiques européennes comme la Bibliothèque Nationale de France (BnF) avec Gallica\footnote{\url{https://gallica.bnf.fr}} ou celle du Portugal avec \textit{La Biblioteca Nacional Digital}\footnote{\url{https://bndigital.bnportugal.gov.pt/}}, mais aussi des chercheur·se·s et/ou amateurs·rices indépendants comme Michael Hart à l'origine du projet Gutenberg\footnote{\url{https://www.gutenberg.org/}}, ont mené des campagnes de numérisation et de publication d’œuvres littéraires sur le web et mis à disposition des chercheur·se·s de vastes corpus de littératures dans plusieurs langues. Si ces initiatives rendent l'accès aux textes de plus en plus aisé, force est de constater que les outils libres de reconnaissance optique de caractères (ou \textit{optical character recognition}, OCR), utilisés régulièrement dans la communauté HN, génèrent des erreurs et du bruit. Le bruit désigne toutes les erreurs produites par le système de ROC : insertion, suppression, mais aussi substitution d'un ou plusieurs caractères. Le bruit dans les sorties de ROC peut être provoqué par divers éléments sur la page à transcrire : des taches, du texte disposé sur deux colonnes, l'emploi de certaines polices typographiques, etc.
Par ailleurs, une grande majorité des outils de TAL utilisés en aval de la ROC sont entraînés sur des données standards (non bruitées). Ainsi, les chercheur·se·s, notamment en littérature, qui souhaitent utiliser ces outils sur leurs données en conditions réelles sont confronté·e·s à la difficulté d'utiliser des outils informatiques qui ne sont pas toujours adaptés aux données bruitées. 

De fait, les erreurs commises par les systèmes d'extraction d'informations, comme la reconnaissance d'entités nommées (REN ou anglais NER) sont souvent imputées au caractère bruité des sorties de ROC, ce qui induit l'idée que la correction des données en entrée est la seule manière pertinente d'améliorer les résultats de REN. 
Si certains outils de ROC produisent effectivement des erreurs régulières \cite{stanislawek-2019} et qu'il est possible de concevoir un programme de correction automatique à base de règles simples, l'apparition d’erreurs singulières rend difficile la correction. De plus, comme le soulignent \cite{huynh:hal-03034484} et \cite{petkovic:hal-04063970}, s'il est possible d'améliorer les résultats de REN en corrigeant automatiquement les sorties de ROC, celle-ci produit ses propres erreurs. Enfin, à la complexité de la REN à partir de la ROC s'ajoute la variation de la langue employée (diachronique, diatopique) %(français du 17ème siècle \cite{simon_gabay_2020_3826894} et français médiéval, \cite{pinche_2022_7410529})
et la variation du genre (littéraire, critique). La littérature en REN révèle un faible intérêt pour les langues autres que l’anglais (\cite{lejeune:hal-01294127}, \cite{rahimi-etal-2019-massively}), notamment pour des langues moins bien dotées en données numérisées comme le portugais.

Nous souhaitons déterminer si la correction de la ROC, en amont, permet d’améliorer  significativement les résultats de la tâche de REN en aval. Nous proposons tout d'abord une revue de la littérature sur la correction et sur la REN sur des transcriptions de ROC bruitées (section \ref{sec:sota}). Puis nous présentons les corpus littéraires sur lesquels nos analyses s'appuient (section \ref{sec:data}) : une série de textes issus de la Très Grande Bibliothèque (TGB\footnote{\url{http://obvil.lip6.fr/tgb/}}) ainsi que des textes extraits des collections française, anglaise et portugaise de la collection européenne ELTeC -- \textit{European collection of literary texts}\footnote{\url{https://www.distant-reading.net/eltec/}}. La section \ref{sec:expe} présente différentes méthodes d'évaluation manuelles et automatiques des outils de REN \texttt{spaCy}\footnote{\url{https://spacy.io/}} \cite{ines_montani_2023_10009823} et \texttt{stanza}\footnote{\url{https://stanfordnlp.github.io/stanza/ner.html}} \cite{qi2020stanza}, une typologie des contaminations de ROC ayant un impact sur la REN, ainsi que des expérimentations menées avec l’outil de correction automatique JamSpell\footnote{\url{https://github.com/bakwc/JamSpell}} dans la lignée de ce que proposait \cite{petkovic2022impact} et une typologie des erreurs de la correction automatique. Enfin nous exposons nos perspectives d’utilisation des résultats dans la section \ref{sec:concl}.

