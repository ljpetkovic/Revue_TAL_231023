Les techniques de traitement automatique des langues (TAL), combinées aux méthodes des humanités numériques (HN), rendent possibles l'exploration et l'exploitation de corpus numérisés à grande échelle. Ces deux domaines trouvent souvent leurs applications dans l'extraction d'informations (reconnaissance d'entités nommées, REN) d'une part, et d'autre part dans la valorisation des corpus patrimoniaux (reconnaissance optique de caractères, ROC).
Les institutions publiques européennes, internationales et des membres indépendants de la communauté scientifique
% comme la Bibliothèque Nationale de France (BnF) avec Gallica\footnote{\url{https://gallica.bnf.fr}} ou celle du Portugal avec \textit{La Biblioteca Nacional Digital}\footnote{\url{https://bndigital.bnportugal.gov.pt/}},mais  comme Michael Hart à l'origine du projet Gutenberg\footnote{\url{https://www.gutenberg.org/}}, 
ont mené des campagnes de numérisation et de publication des transcriptions d’œuvres littéraires sur le web. Ils ont mis à disposition de la communauté de vastes corpus dont la qualité est hétérogène.
En effet, si ces initiatives rendent l'accès aux textes plus aisé, force est de constater que la ROC génère du bruit. Le bruit désigne toutes les erreurs produites par le système de ROC : insertion, suppression, mais aussi substitution d'un ou plusieurs caractères. Le bruit dans les sorties de la ROC peut être provoqué par des taches, du texte disposé sur deux colonnes, l'emploi de certaines polices typographiques, etc.

% RDV Joseph
% Intro : parler de la position du chercheur en Littéraure pas informaticien
% 3 et 4 protocole expérérimental
% séparer les points 5.1 et 5.2 car ce sont deux questions différentes facile à identifier mise en avant 
% Mettre plus en avant l'utilisateur Lettre pas informatiques, on a des modèles sur l'étagère
% Les résultats statistiques
% etats des lieux de tout ce qui peut se passer dans cette enchainement 
% en formalisant explicitement les questions auxquelles on veut répondre
% être plus clair sur l'effet de seuil, 
% est ce que lévaluation manuelle vient corréler NERVAL
% Il y a un besoin d'outil de correction automatique
% quand on donne les résultats dire quelle hypothèse ca permet d'entrevoir

% RDV DAMIEN
%Grace aux techniques de TAL les chercheurs en SHS arrivent à constituer de plus grand corpus
%l'interet des chercheurs en TAL c'est de pouvoir accéder à leur données
%
%Il y a déjà des erreurs sur la REN à la base.  
%Il y a en plus les erreurs de ROC qui impactent la REN
%Préciser qu'on ne travaille pas sur l'amélioration de la REN
%Intro pédagogique qui explique les outils de chercheurs en SHS, pour poser le problème
%Etat de l'art segmenter plus OCR, puis REN, puis interaction, puis correction.
% version et Réf. rajouter un exemple très détaillé
% On mesure les hapax car c'est une mesure de la contamination
% Bien assumer le fait que on dit que la correction c'est pas forcément bénéfique
% Danger de la contamination par surcorrection

Par ailleurs, une grande majorité des outils de TAL utilisés en aval de la ROC sont entraînés sur des données préparées (non bruitées). Ainsi, les scientifiques des sciences humaines et sociales (SHS) et des HN qui utilisent ces outils sur leurs données en conditions réelles rencontrent des difficultés liées à l'inadaptation des outils aux données bruitées. De fait, les erreurs commises par les systèmes de REN sont souvent imputées au caractère bruité des transcriptions de la ROC, ce qui induit l'idée que la correction des données en entrée est la seule manière pertinente d'améliorer les résultats de la REN. 
S'il est possible de corriger automatiquement des erreurs régulières produites par la ROC, l'apparition d’erreurs singulières rend difficile la correction. De plus, comme le soulignent \citeasnoun{huynh:hal-03034484} et \citeasnoun{petkovic:hal-04063970}, s'il est possible d'améliorer les résultats de la REN en corrigeant automatiquement les sorties de la ROC, celle-ci produit ses propres erreurs. Enfin, à la complexité de la REN à partir de la ROC s'ajoute la variation de la langue employée (diachronique, diatopique) %(français du 17ème siècle \cite{simon_gabay_2020_3826894} et français médiéval, \cite{pinche_2022_7410529})
et la variation du genre (littéraire, critique). L'état de l'art en REN révèle un faible intérêt pour les langues autres que l’anglais \cite{lejeune:hal-01294127,rahimi-etal-2019-massively}, notamment pour des langues moins bien dotées comme le portugais.

%Nous souhaitons déterminer si la correction de la ROC, en amont, permet d’améliorer  significativement les résultats de la REN en aval. Nous proposons tout d'abord état de l'art sur la correction et sur la REN sur des transcriptions de ROC bruitées (section \ref{sec:sota}). Puis nous présentons les corpus littéraires sur lesquels nos analyses s'appuient (section \ref{sec:data}) : une série de textes issus de la Très Grande Bibliothèque (TGB\footnote{\url{http://obvil.lip6.fr/tgb/}}) ainsi que des textes extraits des collections française, anglaise et portugaise de la collection européenne ELTeC -- \textit{European collection of literary texts}\footnote{\url{https://www.distant-reading.net/eltec/}}. La section \ref{sec:OCR-IMPACT-NER} présente différentes méthodes d'évaluation manuelles et automatiques de l'outil de REN \texttt{spaCy}\footnote{\url{https://spacy.io/}} \cite{ines_montani_2023_7715077}\footnote{Nous avons effectué les évaluations pour \texttt{stanza}\cite{qi2020stanza} version 1.5.0, vous trouverez les résultats des expériences menées sur le dépôt GitHub : \url{anonymous}.} ; une typologie des contaminations\footnote{\textbf{Nous adoptons le terme contamination proposé par \cite{hamdi:hal-03615997} pour qualifier les entités dont l'orthographe a été modifiée à cause de la transcription fautive de la ROC.}} de ROC ayant un impact sur la REN et une typologie des erreurs de la correction automatique ; des expérimentations menées avec l’outil de correction automatique JamSpell\footnote{Nous avons utilisé la version la plus récente 0.0.12. \url{https://github.com/bakwc/JamSpell}} dans la lignée de ce que proposait \cite{petkovic2022impact}. Enfin nous exposons nos perspectives d’utilisation des résultats dans la section \ref{sec:concl}. La section \ref{sec:OCR-IMPACT-NER} concerne les méthodes d'évaluation de l'impact des contaminations de ROC sur la REN, ainsi que de l'impact des corrections de ROC sur la REN(\ref{sec:COR-OCR-IMPACT-NER}). Nous nous concentrons en particulier sur les évaluations (i) manuelle, (ii) \og{}stricte\fg{} à l'aide des intersections et (iii) \og{}souple\fg{} à l'aide des distances de similarité.
%
%
%section \ref{sec:COR-OCR-IMPACT-NER}

Nous souhaitons déterminer si la correction de la ROC, en amont, permet d’améliorer  significativement les résultats de la REN en aval. Nous proposons en section \ref{sec:sota} un état de l'art portant sur la correction de la ROC et sur la REN à partir des transcriptions bruitées. Puis, en section \ref{sec:data}, nous présentons les corpus littéraires (TGB\footnote{\url{http://obvil.lip6.fr/tgb/}} et ELTeC\footnote{\url{https://www.distant-reading.net/eltec/}}) sur lesquels nos analyses s'appuient. La section \ref{sec:OCR-IMPACT-NER} présente différentes méthodes d'évaluation manuelles et automatiques de l'impact des contaminations\footnote{Nous adoptons le terme \og{}contamination\fg{} proposé par \citeasnoun{hamdi:hal-03615997} pour qualifier les entités dont l'orthographe a été modifiée à cause de la transcription fautive de la ROC.} de la ROC sur la REN effectuée avec l'outil \texttt{spaCy}\footnote{\url{https://spacy.io/}} \cite{ines_montani_2023_7715077}, ainsi qu'une typologie des contaminations. La section \ref{sec:COR-OCR-IMPACT-NER} comprend les évaluations manuelles et automatiques de la REN sur des corrections de la ROC produites avec JamSpell\footnote{\url{https://github.com/bakwc/JamSpell}} (outil de correction automatique), et une typologie des contaminations de la correction de la ROC sur la REN. 
%nous nous concentrons en particulier sur les évaluations (i) manuelle, (ii) \og{}stricte\fg{} à l'aide des intersections et (iii) \og{}souple\fg{} à l'aide des distances de similarité. 
Enfin, nous exposons nos conclusions et les pistes de recherches dans la section \ref{sec:concl}.
